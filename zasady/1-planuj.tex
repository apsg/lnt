
\chapterimage{zasady/1/okladka} % Chapter heading image
\chapter{Planuj i przygotuj się odpowiednio}

\label{rule1-plan-ahead}

Pierwsza zasada LNT skupia się na planowaniu i przygotowaniu. Im lepiej zaplanujemy naszą wyprawę, tym mniej niespodzianek spotka nas po drodze. Harcerze szczycą się czasem umiejętnością radzenia sobie w każdej sytuacji, ale myślę, że prawdziwym powodem do dumy może być takie zaplanowanie swojej aktywności, by być przygotowanym na każdą ewentualność. Każda konieczność improwizacji i ,,radzenia sobie'' pociąga za sobą ryzyko wywarcia destruktywnego wpływu (np. zgubiliśmy się, ale wiemy, że na zachód od nas powinna być droga, więc idziemy w tamtym kierunku ,,na szagę'' nie zważając na zniszczenia po drodze).
Przed wyprawą, biwakiem, obozem, wycieczką, należy zapoznać się z terenem i zaplanować trasę oraz dobrać ekwipunek odpowiednio. Pamiętać należy również o warunkach atmosferycznych. 

Planując cokolwiek warto odpowiedzieć sobie na kilka pytań:
\begin{itemize}
 \item Jaką trasą będziemy podróżować? Czy są tam drogi lub szlaki turystyczne?
 \item Co będziemy jeść? Jeśli wymaga to przygotowania, to gdzie i w jaki sposób?
 \item Co będziemy pić? Czy mam dostatecznie dużo płynów ze sobą, by nie musieć kupować nic w jednorazowych butelkach? Gdzie mogę uzupełnić zapasy?
 \item Jakie są warunki atmosferyczne, czy jesteśmy na nie przygotowani? Czy jesteśmy przygotowani na nagłą zmianę warunków adekwatną do pory roku?
 \item Czy wiem jakie zwierzęta i rośliny napotkamy po drodze? Które są chronione? A które niebezpieczne?
 \item 
\end{itemize}

\begin{zasada}
	Zapoznaj się z przepisami oraz uwarunkowaniami terenu, w jaki się wybierasz.
\end{zasada}

Przygotuj się na nagłe zmiany pogody, zagrożenia i sytuacje awaryjne.
Zaplanuj swoją wyprawę tak, by unikać okresów wzmożonego ruchu turystycznego.
Podróżuj w mniejszych grupach. Rozdziel duże grupy na mniejsze zespoły. 
Przepakuj jedzenie by zminimalizować ilość śmieci.
Używaj mapy i kompasu by wyeliminować konieczność oznaczania trasy np. za pomocą farb czy flag.
